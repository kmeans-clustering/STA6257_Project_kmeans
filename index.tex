% Options for packages loaded elsewhere
\PassOptionsToPackage{unicode}{hyperref}
\PassOptionsToPackage{hyphens}{url}
%
\documentclass[
]{article}
\usepackage{amsmath,amssymb}
\usepackage{lmodern}
\usepackage{iftex}
\ifPDFTeX
  \usepackage[T1]{fontenc}
  \usepackage[utf8]{inputenc}
  \usepackage{textcomp} % provide euro and other symbols
\else % if luatex or xetex
  \usepackage{unicode-math}
  \defaultfontfeatures{Scale=MatchLowercase}
  \defaultfontfeatures[\rmfamily]{Ligatures=TeX,Scale=1}
\fi
% Use upquote if available, for straight quotes in verbatim environments
\IfFileExists{upquote.sty}{\usepackage{upquote}}{}
\IfFileExists{microtype.sty}{% use microtype if available
  \usepackage[]{microtype}
  \UseMicrotypeSet[protrusion]{basicmath} % disable protrusion for tt fonts
}{}
\usepackage{xcolor}
\usepackage[margin=1in]{geometry}
\usepackage{color}
\usepackage{fancyvrb}
\newcommand{\VerbBar}{|}
\newcommand{\VERB}{\Verb[commandchars=\\\{\}]}
\DefineVerbatimEnvironment{Highlighting}{Verbatim}{commandchars=\\\{\}}
% Add ',fontsize=\small' for more characters per line
\usepackage{framed}
\definecolor{shadecolor}{RGB}{248,248,248}
\newenvironment{Shaded}{\begin{snugshade}}{\end{snugshade}}
\newcommand{\AlertTok}[1]{\textcolor[rgb]{0.94,0.16,0.16}{#1}}
\newcommand{\AnnotationTok}[1]{\textcolor[rgb]{0.56,0.35,0.01}{\textbf{\textit{#1}}}}
\newcommand{\AttributeTok}[1]{\textcolor[rgb]{0.77,0.63,0.00}{#1}}
\newcommand{\BaseNTok}[1]{\textcolor[rgb]{0.00,0.00,0.81}{#1}}
\newcommand{\BuiltInTok}[1]{#1}
\newcommand{\CharTok}[1]{\textcolor[rgb]{0.31,0.60,0.02}{#1}}
\newcommand{\CommentTok}[1]{\textcolor[rgb]{0.56,0.35,0.01}{\textit{#1}}}
\newcommand{\CommentVarTok}[1]{\textcolor[rgb]{0.56,0.35,0.01}{\textbf{\textit{#1}}}}
\newcommand{\ConstantTok}[1]{\textcolor[rgb]{0.00,0.00,0.00}{#1}}
\newcommand{\ControlFlowTok}[1]{\textcolor[rgb]{0.13,0.29,0.53}{\textbf{#1}}}
\newcommand{\DataTypeTok}[1]{\textcolor[rgb]{0.13,0.29,0.53}{#1}}
\newcommand{\DecValTok}[1]{\textcolor[rgb]{0.00,0.00,0.81}{#1}}
\newcommand{\DocumentationTok}[1]{\textcolor[rgb]{0.56,0.35,0.01}{\textbf{\textit{#1}}}}
\newcommand{\ErrorTok}[1]{\textcolor[rgb]{0.64,0.00,0.00}{\textbf{#1}}}
\newcommand{\ExtensionTok}[1]{#1}
\newcommand{\FloatTok}[1]{\textcolor[rgb]{0.00,0.00,0.81}{#1}}
\newcommand{\FunctionTok}[1]{\textcolor[rgb]{0.00,0.00,0.00}{#1}}
\newcommand{\ImportTok}[1]{#1}
\newcommand{\InformationTok}[1]{\textcolor[rgb]{0.56,0.35,0.01}{\textbf{\textit{#1}}}}
\newcommand{\KeywordTok}[1]{\textcolor[rgb]{0.13,0.29,0.53}{\textbf{#1}}}
\newcommand{\NormalTok}[1]{#1}
\newcommand{\OperatorTok}[1]{\textcolor[rgb]{0.81,0.36,0.00}{\textbf{#1}}}
\newcommand{\OtherTok}[1]{\textcolor[rgb]{0.56,0.35,0.01}{#1}}
\newcommand{\PreprocessorTok}[1]{\textcolor[rgb]{0.56,0.35,0.01}{\textit{#1}}}
\newcommand{\RegionMarkerTok}[1]{#1}
\newcommand{\SpecialCharTok}[1]{\textcolor[rgb]{0.00,0.00,0.00}{#1}}
\newcommand{\SpecialStringTok}[1]{\textcolor[rgb]{0.31,0.60,0.02}{#1}}
\newcommand{\StringTok}[1]{\textcolor[rgb]{0.31,0.60,0.02}{#1}}
\newcommand{\VariableTok}[1]{\textcolor[rgb]{0.00,0.00,0.00}{#1}}
\newcommand{\VerbatimStringTok}[1]{\textcolor[rgb]{0.31,0.60,0.02}{#1}}
\newcommand{\WarningTok}[1]{\textcolor[rgb]{0.56,0.35,0.01}{\textbf{\textit{#1}}}}
\usepackage{graphicx}
\makeatletter
\def\maxwidth{\ifdim\Gin@nat@width>\linewidth\linewidth\else\Gin@nat@width\fi}
\def\maxheight{\ifdim\Gin@nat@height>\textheight\textheight\else\Gin@nat@height\fi}
\makeatother
% Scale images if necessary, so that they will not overflow the page
% margins by default, and it is still possible to overwrite the defaults
% using explicit options in \includegraphics[width, height, ...]{}
\setkeys{Gin}{width=\maxwidth,height=\maxheight,keepaspectratio}
% Set default figure placement to htbp
\makeatletter
\def\fps@figure{htbp}
\makeatother
\setlength{\emergencystretch}{3em} % prevent overfull lines
\providecommand{\tightlist}{%
  \setlength{\itemsep}{0pt}\setlength{\parskip}{0pt}}
\setcounter{secnumdepth}{5}
\newlength{\cslhangindent}
\setlength{\cslhangindent}{1.5em}
\newlength{\csllabelwidth}
\setlength{\csllabelwidth}{3em}
\newlength{\cslentryspacingunit} % times entry-spacing
\setlength{\cslentryspacingunit}{\parskip}
\newenvironment{CSLReferences}[2] % #1 hanging-ident, #2 entry spacing
 {% don't indent paragraphs
  \setlength{\parindent}{0pt}
  % turn on hanging indent if param 1 is 1
  \ifodd #1
  \let\oldpar\par
  \def\par{\hangindent=\cslhangindent\oldpar}
  \fi
  % set entry spacing
  \setlength{\parskip}{#2\cslentryspacingunit}
 }%
 {}
\usepackage{calc}
\newcommand{\CSLBlock}[1]{#1\hfill\break}
\newcommand{\CSLLeftMargin}[1]{\parbox[t]{\csllabelwidth}{#1}}
\newcommand{\CSLRightInline}[1]{\parbox[t]{\linewidth - \csllabelwidth}{#1}\break}
\newcommand{\CSLIndent}[1]{\hspace{\cslhangindent}#1}
\usepackage{booktabs}
\usepackage{longtable}
\usepackage{array}
\usepackage{multirow}
\usepackage{wrapfig}
\usepackage{float}
\usepackage{colortbl}
\usepackage{pdflscape}
\usepackage{tabu}
\usepackage{threeparttable}
\usepackage{threeparttablex}
\usepackage[normalem]{ulem}
\usepackage{makecell}
\usepackage{xcolor}
\ifLuaTeX
  \usepackage{selnolig}  % disable illegal ligatures
\fi
\IfFileExists{bookmark.sty}{\usepackage{bookmark}}{\usepackage{hyperref}}
\IfFileExists{xurl.sty}{\usepackage{xurl}}{} % add URL line breaks if available
\urlstyle{same} % disable monospaced font for URLs
\hypersetup{
  pdftitle={K-Means Clustering},
  pdfauthor={Anand Pandey; Katie Hidden; Akash Chandra},
  hidelinks,
  pdfcreator={LaTeX via pandoc}}

\title{K-Means Clustering}
\author{Anand Pandey \and Katie Hidden \and Akash Chandra}
\date{2022-11-05}

\begin{document}
\maketitle

\begin{center}\rule{0.5\linewidth}{0.5pt}\end{center}

\hypertarget{introduction}{%
\section{Introduction}\label{introduction}}

We have often heard of training machine learning models with labeled
data. Imaging if there is a massive volume of data with no labels and we
want to come up with a scalable approach to process these data and find
insights. Difficult as it may seem, this is possible with clustering
algorithms like K-Means.

Unsupervised machine learning is a type of algorithm that works on
detecting patterns from a dataset when outcomes are not known or
labeled. In unsupervised learning models it is not possible to train the
algorithm the way we would normally do in case of supervised learning.
This is because the data is neither classified nor labeled and allows
the algorithm to act on the information without supervision. An
unsupervised algorithm works on discovering the underlying hidden
structure, pattern or association of the data and that helps the model
in clustering or grouping the data without any human intervention.

The main goal of this paper is to discuss the concept and underlying
methodology of one the unsupervised algorithm called K-Means clustering.
We will also discuss how K-Means can be leveraged in real time
applications like customer segmentation (Yulin 2020). In this paper, we
will also be discussing various limitations and bottlenecks of K-Means
clustering algorithm and would suggest some improved algorithms to
overcome these limitations.

\hypertarget{k-means-clustering}{%
\subsection{K-Means Clustering}\label{k-means-clustering}}

K-means clustering is used for grouping similar observations together by
minimizing the Euclidean distance between them. It uses ``centroids''.
Initially, it randomly chooses K different points in the data and
assigns every data point to their nearest centroid. Once all of them are
assigned, it moves the centroid to the average of points assigned to it.
When the assigned centroid stops changing, we get the converged data
points in separate clusters.

There are some limitations of K-means clustering. It can be difficult to
determine an appropriate initial K-value, especially with large and
multidimensional datasets. The algorithm is sensitive to the initial
centroid values and may fall into the local optimum solution. K-mean
clustering typically classifies spherically shaped data well, but is
less successful at classifying irregularly shaped data.

\hypertarget{customer-segmentation}{%
\subsection{Customer Segmentation}\label{customer-segmentation}}

Customer segmentation helps divide customers into different groups based
on their common set of characteristics (like age, gender, spending
habit, credit score, etc.) that helps in targeting those customers for
marketing purposes. The primary focus of customer segmentation is to
come up with strategies that helps in identifying customers in each
category in order to maximize the profit by optimizing the services and
products. Therefore, customer segmentation helps businesses in promoting
the right product to the right customer to increase profits(Tabianan
2022).

Customer segmentation is not only helpful for business but also helps
customers by providing them information relevant to their needs. If
customers receive too much information which is not related to their
regular purchase or their interest on the products, it can cause
confusion on deciding their needs. This might lead their customers to
give up on purchasing the items they required and effect the business to
lose their potential customers. The clustering analysis will help to
categorize the customer according to their spending habit, purchase
habit or specific product or brand the customers interested in. Customer
segmentation can be broadly divided into four factors - demographic
psychographic, behavioral, and geographic(Tabianan 2022). In this paper,
customer behavioral factor has been primarily focused.

K-Means clustering algorithm can help effectively extract groups of
customers with similar characteristics and purchasing behavior which in
turn helps businesses to specify their differentiated marketing campaign
and become more customer-centric.(Yulin 2020)

\hypertarget{methods}{%
\section{Methods}\label{methods}}

\hypertarget{analysis-and-results}{%
\section{Analysis and Results}\label{analysis-and-results}}

\hypertarget{dataset-description}{%
\subsection{Dataset Description}\label{dataset-description}}

For the customer segmentation, we will be using a data set that contains
all the transactions that has occurred for a UK-based non-store online
retail between 01/12/2009 and 09/12/2011. The company mainly sells
unique all-occasion gift-ware. Many customers of the company are
wholesalers. The company was established in 80s as a storefront and
relied on direct mailing catalogues and taken order over phone. In
recent years, the company launched a website and shifted completely to a
web based online retail to take technological advantage of
customer-centric targeted marketing approach(Chen 2010).

The customer transactions dataset has 8 variables as shown in below data
definition table.

\begin{table}

\caption{\label{tab:unnamed-chunk-2}Data Definition}
\centering
\begin{tabu} to \linewidth {>{\raggedright}X>{\centering}X>{\raggedright}X}
\hline
\cellcolor[HTML]{666666}{\textcolor{white}{\textbf{Attribute}}} & \cellcolor[HTML]{666666}{\textcolor{white}{\textbf{Type}}} & \cellcolor[HTML]{666666}{\textcolor{white}{\textbf{Description}}}\\
\hline
Invoice & Nominal & Invoice number. A 6-digit integral number uniquely assigned to each transaction. If this code starts with the letter `c`, it indicates a cancellation.\\
\hline
StockCode & Nominal & Product (item) code. A 5-digit integral number uniquely assigned to each distinct product.\\
\hline
Description & Nominal & Product (item) name.\\
\hline
Quantity & Numeric & The quantities of each product (item) per transaction.\\
\hline
InvoiceDate & Numeric & Invice date and time. The day and time when a transaction was generated.\\
\hline
Price & Numeric & Product price per unit in sterling (£).\\
\hline
CustomerID & Numeric & Customer number. A 5-digit integral number uniquely assigned to each customer.\\
\hline
Country & Nominal & Country name. The name of the country where a customer resides.\\
\hline
\end{tabu}
\end{table}

\hypertarget{data-preparation}{%
\subsection{Data Preparation}\label{data-preparation}}

Before we prepare the data, let's take a look at the summary and see if
there are any missing values or other trends.

\begin{table}

\caption{\label{tab:unnamed-chunk-5}Summary of Data}
\centering
\begin{tabu} to \linewidth {>{\raggedright}X>{\centering}X>{\centering}X>{\raggedright}X>{\centering}X>{\centering}X>{\raggedright}X>{\centering}X>{\centering}X>{\raggedright}X}
\hline
\cellcolor[HTML]{666666}{\textcolor{white}{\textbf{}}} & \cellcolor[HTML]{666666}{\textcolor{white}{\textbf{N}}} & \cellcolor[HTML]{666666}{\textcolor{white}{\textbf{Missing}}} & \cellcolor[HTML]{666666}{\textcolor{white}{\textbf{Mean}}} & \cellcolor[HTML]{666666}{\textcolor{white}{\textbf{SD}}} & \cellcolor[HTML]{666666}{\textcolor{white}{\textbf{Min}}} & \cellcolor[HTML]{666666}{\textcolor{white}{\textbf{Q1}}} & \cellcolor[HTML]{666666}{\textcolor{white}{\textbf{Median}}} & \cellcolor[HTML]{666666}{\textcolor{white}{\textbf{Q3}}} & \cellcolor[HTML]{666666}{\textcolor{white}{\textbf{Max}}}\\
\hline
Quantity & 525461 & 0 & 10.34 & 107.42 & -9600.00 & 1.00 & 3.0 & 10.00 & 19152.00\\
\hline
Price & 525461 & 0 & 4.69 & 146.13 & -53594.36 & 1.25 & 2.1 & 4.21 & 25111.09\\
\hline
Customer Id & 417534 & 107927 & - & - & - & - & - & - & -\\
\hline
\end{tabu}
\end{table}

The CustomerID contains unique Id for each unique Customer. However, we
have 107,927 customers with no customer ids. We will be removing these
customer before starting our analysis.

\begin{Shaded}
\begin{Highlighting}[]
\NormalTok{df}\OtherTok{=}\FunctionTok{na.omit}\NormalTok{(df, }\AttributeTok{cols=}\StringTok{"CustomerID"}\NormalTok{)}
\end{Highlighting}
\end{Shaded}

\hypertarget{data-and-vizualisation}{%
\subsection{Data and Vizualisation}\label{data-and-vizualisation}}

\hypertarget{statistical-modeling}{%
\subsection{Statistical Modeling}\label{statistical-modeling}}

\hypertarget{conlusion}{%
\section{Conlusion}\label{conlusion}}

\hypertarget{references}{%
\section*{References}\label{references}}
\addcontentsline{toc}{section}{References}

\hypertarget{refs}{}
\begin{CSLReferences}{1}{0}
\leavevmode\vadjust pre{\hypertarget{ref-online2010Chen}{}}%
Chen, Dr. Daqing. 2010. {``Online Retail II Data Set.''} School of
Engineering, London South Bank University, London SE1 0AA, UK.
\url{https://archive.ics.uci.edu/ml/datasets/Online+Retail+II}.

\leavevmode\vadjust pre{\hypertarget{ref-tab2022kmeans}{}}%
Tabianan, S., Velu. 2022. {``K-Means Clustering Approach for Intelligent
Customer Segmentation Using Customer Purchase Behavior Data.''}
\emph{Analytical Methods} 14 (12): 7243.
\url{https://doi.org/10.3390/su14127243}.

\leavevmode\vadjust pre{\hypertarget{ref-yul2020astudy}{}}%
Yulin, \& Qianying, D. 2020. {``A Study on e-Commerce Customer
Segmentation Management Based on Improved k-Means Algorithm.''}
\emph{Information Systems and e-Business Management} 18 (4): 497--510.
\url{https://doi.org/10.1007/s10257-018-0381-3}.

\end{CSLReferences}

\end{document}
